\documentclass{IEEEtran}
\usepackage{amsmath}
\usepackage{graphicx}
%date{29 April,2016}
%\date{\today}
\author{Gaurav,Manas,Sahaj}
\title{Sorting Algorithms}

\begin{document}
\maketitle


\section{Selection sort}
\begin{flushleft}
The algorithm divides the input list into two parts: \newline
the sublist of items already sorted, which is built up from left to right at the front (left) of the list, and the sublist of items remaining to be sorted that occupy the rest of the list. \newline Initially, the sorted sublist is empty and the unsorted sublist is the entire input list. \newline The algorithm proceeds by finding the smallest (or largest, depending on sorting order) element in the unsorted sublist, exchanging (swapping) it with the leftmost unsorted element (putting it in sorted order), and moving the sublist boundaries one element to the right.
\end{flushleft}

\section{Merge sort}
\begin{flushleft}
Conceptually, a merge sort works as follows:
\begin{itemize}
\item Divide the unsorted list into n sublists, each containing 1 element (a list of 1 element is considered sorted).
\item Repeatedly merge sublists to produce new sorted sublists until there is only 1 sublist remaining. This will be the sorted list.
\end{itemize}
\end{flushleft}

\section{Quick sort}
\begin{flushleft}
Quicksort is a divide and conquer algorithm. Quicksort first divides a large array into two smaller sub-arrays: the low elements and the high elements. Quicksort can then recursively sort the sub-arrays. \newline
\end{flushleft}

\section{Bubble sort}
\begin{flushleft}
Bubble sort, sometimes referred to as sinking sort, is a simple sorting algorithm that repeatedly steps through the list to be sorted, compares each pair of adjacent items and swaps them if they are in the wrong order. The pass through the list is repeated until no swaps are needed, which indicates that the list is sorted. 
\end{flushleft}

\section{Insertion Sort}
\begin{flushleft}
Insertion sort iterates, consuming one input element each repetition, and growing a sorted output list. Each iteration, insertion sort removes one element from the input data, finds the location it belongs within the sorted list, and inserts it there. It repeats until no input elements remain.\newline \newline \newline
\end{flushleft}

\section{Conclusion}
\begin{flushleft}
After running our sorting algorithms several times for different array size ranging from 10 to 1,000,000 \newline
We came to the following conclusion: \newline
\begin{itemize}
\item Merge sort is the fastest as it takes lesser number of comparisons in all the cases. It also means that it is efficient for complicated problems.
\item Selection sort is the slowest and possibly an impractical way of sorting as it takes huge number of comparison.
\end{itemize}
\end{flushleft}

\begin{figure}
\includegraphics[width=3in]{/home/gaurav/Desktop/plottime.eps}
\caption{No. of comparisons}
\end{figure}

\begin{figure}
\includegraphics[width=3in]{/home/gaurav/Desktop/plotcomp.eps}
\caption{No. of comparisons}
\end{figure}

\end{document}
